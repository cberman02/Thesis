\documentclass [11pt]{article}
\usepackage{booktabs}
\usepackage{geometry}
\usepackage{graphicx}
\usepackage{amsmath}
\usepackage{dcolumn}
\usepackage{float}
\restylefloat{table}

% KGC: Define acronyms like TNC in full when you introduce them.

% KGC: Also, I believe you mean complement here. Compliment is a nice thing to say to someone. Complement is a nice way to describe two things that go well together.

\title{Transportation Network Company Integration and Subsidization as a Complement to Public Transportation}
\author{Charlie Berman}

\begin{document}
\maketitle

\section*{Introduction}

% KGC: Try a more active intro hook that sets up what you're going to talk about. The example I provide below is imperfect, but it is a start. 

%"In many urban areas, only a fraction of residents live a short walk away from a public transit stop, making public transportation a less convenient option for most (cite if you have one). Named the first mile/last mile problem by (cite), this inconvenience often outweighs the advantages of a modern public transportation system leading commuters to favor individual cars and transit ridership to decline (cite here). Since 2012, bus ridership and rail ridership have fallen 15\% and 3\%, respectively, which Erhardt et al. (2022) attribute to increased car ownership, lower gas prices, higher fares and incomes, and a rise in remote work. More recently, ridership cratered during the COVID-19 pandemic, and has not yet fully rebounded to pre-pandemic levels. 

The benefits of a modern public transportation system are fairly well-documented. Proponents claim that public transit represents a cheaper and more sustainable mode of transportation, as commuters rely less on individual cars. Use of public transit has also been linked to a healthier population, as well as being heavily relied on by young and low-income populations (Heaps, 2021). Despite the obvious social benefits, transit agencies nationwide have faced declining ridership since 2012, with one study estimating a bus ridership decrease of 15\% and a rail ridership decrease of 3\% (Erhardt, et al. 2022). This same study attributed this drop partially to higher fares and incomes, remote work, lower gas prices and a rise in car ownership. More recently, ridership cratered during the COVID-19 pandemic, and has not yet fully rebounded to pre-pandemic levels. \\
%KGC: Fine to just say it has not rebounded yet, this implies it has recovered somewhat but not completely. 


% KGC: Here you want to transition into more active language again: 

One potential solution to the first mile/last mile problem is the integration of transportation network companies (TNCs) like Uber and Lyft into public transit systems. TNCs have been a popular mode of transportation since their inception, and have been used to fill in the gaps of public transit systems. In 2014, Uber launched its lower priced UberX service, and Lyft expanded into a total of 60 cities that same year. This paper addresses the potential for TNCs to be integrated into public transit systems and the impact of such integration on ridership.

% KGC: Then dive into the decline in ridership overlapping with TNCs because it is contradictory to the above proposal. 

% KGC: I've added suggestions below to help you be a bit snappier. In general, you can be more direct and active in your writing.

\indent This decline from 2012-2018 happens to coincide neatly with the rise in popularity of transport network companies (TNC) like Uber and Lyft. With the launch of the lower-priced UberX service and Lyft's expansion into 60 cities in 2014, it seems plausible that these companies played a role in decreasing ridership, but urban simulation models suggest a more complex relationship. 

In a monocentric city model, Zhao (2019) finds that TNCs are complements for "high quality" public transportation systems and substitutes for "low quality" systems in the long run. Zhao and Agrawal (2023) extend this monocentric city model, basing it off Chicago and allowing for multiple transportation modes, to simulate multiple policy interventions. They find that public transit ridership increases more when TNCs are subsidized to aid in the first mile/last mile problem, while taxes lead to a minimal boost in ridership, suggesting that policy determines whether TNCs are substitutes or complements. 

While there is potential for TNCs to boost public transit ridership, empirical evidence suggests that potential is often unrealized and may be unattainable. Erhardt, et al. (2022) claims that TNCs are the biggest driver of decreasing public transit ridership after examining ridership data from the National Transit Database, or NTD. (KGC: What method?) On the other hand, Hall et al. (2018) measure Uber penetration (KGC: put this in layman's terms, are they just doing diff-in-diffs around entry?) and use a difference-in-differences design to find that Uber is on average a complement to public transit (so public transit increasing with Uber?), though there is considerable heterogeneity in ridership, suggesting TNCs can be a substitute or a complement. 

\indent Given the conflicting evidence, many public transit agencies have gambled on improving ridership through a TNC partnership. These agencies have started pilot programs around 2016-17 in the hopes of combining the convenience of ridesharing with the low cost and ability to cover ground of public transit. There are several strategies that agencies have used to partner with TNCs. Agencies in cities like Dallas and Atlanta have integrated a ride hailing option into their transit app, making it easier for riders to summon a ride from a public transit station. Other agencies, like in Summit, NJ, have subsidized demand for TNCs, either by offering free or discounted rides to public transit stations. (KGC: It seems like you list multiple types of subsidies here, but they're all just forms of subsidizing demand.) Other agencies have used TNCs to aid their paratransit (KGC: define!) programs to free up budget and increase ridership. (KGC: Unclear here -- do they just have Uber provide rides for people with disabilities?)\\

% KGC: Move the bolded part up. This is what you plan to do. Then tell me about the antecedents and the "value added." The value added is the gap in the literature that you are filling -- if that is the question state it. But if you are not the first to ask this question about TNCs, then that is not a value add. 

\indent Given the relative recency of these programs, there is very limited research on whether these programs will succeed in increasing ridership. This paper aims to be one of the first to examine the outcomes of these programs. While Agrawal, et al. (2023) come closest, their work relies on a monocentric model as opposed to actual ridership data, which this paper will use. \textbf{This paper seeks to answer the efficacy of increasing public transit ridership through a partnership with TNCs.} 
There are two partnerships being examined: agencies who make it easier to summon a ride through their own app, and agencies who actively subsidize TNC rides. \textit{I anticipate that subsidies will have a negligible impact on public transit ridership due to the subsidies likely being funded through an increase in taxes, rendering any savings moot. I expect that including the ability to summon a ride from a public transit app will lead to a statistically significant, yet small rise in ridership as it will increase the likelihood of older, less technically savvy users of using a TNC.} \\

%KGC: I cut down a bit, but this is a decent explanation. One suggestion: do a TWFE event study regression where you look at the effect of taking any policy action on ridership. This will help you to see if the policy is effective and if it is effective, when it is effective. Then split into the two groups. 

\indent One of the most comprehensive partnerships comes from Pinellas County in Florida. After trial programs, Pinellas Suncoast Transit Authority, or PSTA, created the Direct Connect Program in January 2017. This program gives users \$5 off their trip if they are going to or from one of 26 different bus stop zones throughout the region. This discount is billed as a way to partially eliminate the first mile/last mile problem and get users to bus stops more easily. Using a difference-in-difference specification, I compare the outcome variable of ridership before and after the introduction of the Direct Connect Program and compare that to ridership across other Florida transit agencies. I will apply the same methodology to cities like Dallas, which feature app integration but no subsidies, in the hopes of figuring out what programs are most effective. Significant results in the interaction term of treatment and post will indicate the efficacy of the program. I expect that there will be significant, yet fairly low increase in ridership for both strategies. Other variables in the regression specification will include demographic and socioeconomic variables in order to determine if these changes play a role in ridership, as well as a way to determine similar cities to the treated group. By finding similar locations that are untreated, sources of variance will hopefully be reduced.\\


% KGC: When you shift from the future to present tense, this paragraph will flow better.

\indent In order to conduct this regression, I rely on a panel dataset compiled from the Federal Transit Administration's National Transit Database (NTD) and the US Census's American Community Survey (ACS). The NTD dataset collects monthly ridership data from every transit agency that receives federal funding. This is merged with ACS data on race, median income, college education, and population at the urbanized area (UZA) level for every year. To qualify as a UZA, the population must be greater than 65,000, so some areas will not be represented in the data. The dataset spans 2005-2019, excluding the COVID years, for 68 unique transit agencies from all 50 states and Washington DC, which make up the 295,000 observations. Some agencies are not active for the entire timeframe, so consideration must be given to whether they should remain in the dataset or not. Other potential sources of error could come from ACS data being reported yearly, as opposed to the monthly reporting of NTD data.\\

% KGC: In the future, write this section in the present instead of future tense.

\indent This paper will first discuss in detail the different strategies that agencies use when they partner with a TNC. The two main strategies are largely the app integration and the app integration plus subsidized rides to public transit stations. The regression model will then be discussed, with an emphasis on defining the treatment and control groups, as well as a discussion of the methodology in choosing them. These will be subset from the data in such a way to minimize variance in UZA characteristics between the two groups. The difference-in-difference regression will then be ran for the two major strategies, where the interaction term will be studied for significant results and in what way they influence ridership. Robustness checks will then be performed to check the validity of the results. I will use the results of this analysis to discuss potential policy implications and how agencies should handle these partnerships going forward.

% KGC: Overall, great job. I rewrote fairly heavily because it seemed the easiest way to communicate some points in this setting. Free disposal if you like. I think you have a good start here and will learn through the process of writing the paper both how to be an even more effective writer and how to be an even more effective researcher.

\nocite{*}
\newpage
\bibliography{references}
\bibliographystyle{apalike}

\end{document}
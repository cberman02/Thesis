\documentclass [11pt]{article}
\usepackage{booktabs}
\usepackage{geometry}
\usepackage{graphicx}
\usepackage{amsmath}
\usepackage{dcolumn}
\usepackage{float}
\restylefloat{table}

\title{TNC Integration and Subsidization as a Compliment to Public Transportation}
\author{Charlie Berman}

\begin{document}
\maketitle

\section*{Introduction}
The benefits of a modern public transportation system are fairly well-documented. Proponents claim that public transit represents a cheaper and more sustainable mode of transportation, as commuters rely less on individual cars. Use of public transit has also been linked to a healthier population, as well as being heavily relied on by young and low-income populations (Heaps, 2021). Despite the obvious social benefits, transit agencies nationwide have faced declining ridership since 2012, with one study estimating a bus ridership decrease of 15\% and a rail ridership decrease of 3\% (Erhardt, et al. 2022). This same study attributed this drop partially to higher fares and incomes, remote work, lower gas prices and a rise in car ownership. More recently, the COVID-19 pandemic led to cratering in ridership, and while the data suggests that ridership has rebounded somewhat, it has yet to return to pre-pandemic levels.\\
\indent This decline from 2012-2018, ignoring the pandemic years, happens to coincide neatly with the rise in popularity of transport network companies, TNCs, like Uber and Lyft. With Uber's lower priced UberX service launching in 2014, and Lyft expanding into a total of 60 cities that same year, it seems highly plausible that these companies have played a role in decreasing ridership. Indeed, Erhardt, et al. (2022) claims that TNCs are the biggest driver of decreasing public transit ridership after examining ridership data from the National Transit Database, or NTD. There is some ambiguity about whether this is the case, with earlier work from Hall, et al. (2018). They measure Uber penetration and perform a difference-in-differences approach and find that Uber is on average a compliment to public transit. The authors do acknowledge that this is not universal, and that there is considerable heterogeneity in whether transit agencies will see a boost or decline in ridership. A year later, Zhao (2019) uses simulation modeling of a monocentric city to argue that TNCs can compliment a "high quality" public transportation system but substitute a "low quality" system in the long run. Zhao also partners with Agrawal to extend the monocentric city model, basing it off of Chicago, to allow for multiple transportation modes and test multiple policies (2023). Their model finds that public transit ridership increases more when TNCs are subsidized to aid in the first mile/last mile problem more than when they are taxed, which results in a minimal boost in ridership. They also claim that whether TNCs are substitutes or compliments is based on policy. Based on this research, it seems safe to say that there is potential for TNCs to be integrated into public transit based on models, but a data-driven approach suggests that potential may be unrealized or unattainable.\\
\indent Given the conflicting evidence, public transit agencies in multiple states have gambled on being able to improve ridership via a partnership with TNCs. These agencies started pilot programs around 2016-17 in the hopes of combining the convenience of ridesharing with the low cost and ability to cover ground of public transit. How these agencies approach this situation is subject to some variation. Cities like Dallas and Atlanta included the ability to hail a ride from their transit app. Some cities have taken more direct measures, like Summit, NJ offering free Uber rides as an alternative to building a parking garage near a transit station. Agencies have also piloted voucher programs to reduce the burden on riders to use TNCs to get to stations. Others have sought to use TNCs to aid their paratransit programs to free up budget and increase ridership.\\
\indent Given the relative recency of these programs, there is very limited research on whether these programs will succeed in increasing ridership. This paper aims to be one of the first to examine the outcomes of these programs. While Agrawal, et al. (2023) come closest, their work relies on a monocentric model as opposed to actual ridership data, which this paper will use. \textbf{This paper seeks to answer the efficacy of increasing public transit ridership through a partnership with TNCs.} There are two partnerships being examined: agencies who make it easier to summon a ride through their own app, and agencies who actively subsidize TNC rides. \textit{I anticipate that subsidies will have a negligible impact on public transit ridership due to the subsidies likely being funded through an increase in taxes, rendering any savings moot. I expect that including the ability to summon a ride from a public transit app will lead to a statistically significant, yet small rise in ridership as it will increase the likelihood of older, less technically savvy users of using a TNC.} \\
\indent One of the most comprehensive partnerships comes from Pinellas County in Florida. After trial programs, Pinellas Suncoast Transit Authority, or PSTA, created the Direct Connect Program in January 2017. This program gives users \$5 off their trip if they are going to or from one of 26 different bus stop zones throughout the region. This discount is billed as a way to partially eliminate the first mile/last mile problem and get users to bus stops more easily. This program will be one of the main focuses of the efficacy of subsidizing uber. Using a difference-in-difference specification, I compare the outcome variable of ridership before and after the introduction of this program and compare that to ridership across other Florida transit agencies. This same protocol will be applied to cities like Dallas, which feature app integration but no subsidies, in the hopes of figuring out what programs are most effective. Significant results in the interaction term of treatment and post will indicate the efficacy of the program. I expect that there will be significant, yet fairly low increase in ridership for both strategies, for reasons explained above. Other variables in the regression specification will include demographic and socioeconomic variables in order to determine if these changes play a role in ridership, as well as a way to determine similar cities to the treated group. By finding similar locations that are untreated, sources of variance will hopefully be reduced.\\
\indent In order to conduct this regression, I rely on a panel dataset compiled from the Federal Transit Administration's National Transit Database (NTD) and the US Census's American Community Survey (ACS). The NTD dataset collects monthly ridership data from every transit agency that receives federal funding. This is merged with ACS data on race, median income, college education, and population at the urbanized area (UZA) level for every year. To qualify as a UZA, the population must be greater than 65,000, so some areas will not be represented in the data. The dataset spans 2005-2019, excluding the COVID years, for 68 unique transit agencies from all 50 states and Washington DC, which make up the 295,000 observations. Some agencies are not active for the entire timeframe, so consideration must be given to whether they should remain in the dataset or not. Other potential sources of error could come from ACS data being reported yearly, as opposed to the monthly reporting of NTD data.\\
\indent This paper will first discuss in detail the different strategies that agencies use when they partner with a TNC. The two main strategies are largely the app integration and the app integration plus subsidized rides to public transit stations. The regression model will then be discussed, with an emphasis on defining the treatment and control groups, as well as a discussion of the methodology in choosing them. These will be subset from the data in such a way to minimize variance in UZA characteristics between the two groups. The difference-in-difference regression will then be ran for the two major strategies, where the interaction term will be studied for significant results and in what way they influence ridership. Robustness checks will then be performed to check the validity of the results. I will use the results of this analysis to discuss potential policy implications and how agencies should handle these partnerships going forward.


\nocite{*}
\newpage
\bibliography{references}
\bibliographystyle{apalike}

\end{document}